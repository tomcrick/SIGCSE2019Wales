%%%% Proceedings format for most of ACM conferences (with the exceptions listed below) and all ICPS volumes.
\documentclass[sigconf]{acmart}
\usepackage{paralist}
\usepackage{url}
\usepackage[hyphenbreaks]{breakurl}

\def\UrlBreaks{\do\/\do-}

%%%% As of March 2017, [siggraph] is no longer used. Please use sigconf (above) for SIGGRAPH conferences.

%%%% Proceedings format for SIGPLAN conferences 
% \documentclass[sigplan, anonymous, review]{acmart}

%%%% Proceedings format for SIGCHI conferences
% \documentclass[sigchi, review]{acmart}

\usepackage{booktabs} % For formal tables


% Copyright
%\setcopyright{none}
%\setcopyright{acmcopyright}
\setcopyright{acmlicensed}
%\setcopyright{rightsretained}
%\setcopyright{usgov}
%\setcopyright{usgovmixed}
%\setcopyright{cagov}
%\setcopyright{cagovmixed}

\copyrightyear{2019}
\acmYear{2019}
\setcopyright{acmlicensed}
\acmConference[SIGCSE '19]{The 50th ACM Technical Symposium on
  Computer Science Education}{Feb. 27--Mar 2, 2019}{Minneapolis, MN, USA}
%\acmBooktitle{}
%\acmPrice{15.00}
%\acmDOI{10.1145/3159450.3159547}
%\acmISBN{978-1-4503-5103-4/18/02}
% This slight change to the code may also save 1 or 2 lines of space.

% removes the headers from each page per the preparation instructions, as these are not needed and will be updated with the chairs' actual session names during the pagination/indexing process:
\fancyhead{}

\begin{document}
\title{Computing and Digital Competence Within a Radical Curriculum Reform in Wales}
%\titlenote{}
%\subtitle{Extended Abstract}
%\subtitlenote{}

\author{Tom Crick}
\orcid{0000-0001-5196-9389}
\affiliation{%
  \institution{Swansea University}
  \streetaddress{}
  \city{Swansea} 
  \country{United Kingdom}
}
\email{thomas.crick@swansea.ac.uk}

\author{Faron Moller}
\orcid{0000-0001-9535-8053}
\affiliation{%
  \institution{Swansea University}
  \streetaddress{}
  \city{Swansea} 
  \country{United Kingdom}
}
\email{f.g.moller@swansea.ac.uk}
 
% The default list of authors is too long for headers}
%\renewcommand{\shortauthors}{Crick and Moller}


\begin{abstract}
We are witnessing significant computer science curriculum reform
across a number of regions and nations. Such reform in the United
Kingdom has been subject to intense scrutiny, particularly since
England introduced tis new computing curriculum in September 2014. In
Wales, we are starting to see the implementation and potential impact
of a radical new co-constructed, practitioner-led and purpose-driven
curriculum, with digital competencies as a core cross-curricular
responsibility distinct from computer science. As part of this reform,
the new curriculum in Wales is to be organised around six ``areas of
learning and experience'' (AoLEs), with a new Science \& Technology
AoLE bringing together the traditional physical sciences (physics,
chemistry and biology) along with computer science and design \&
technology.

% novel -- no one else is doing this -- cross-curricular digital
% competencies
% first time that any nation has given parity to DC to literacy and
% numeracy
In this paper, we present a developing national case study based
around the new computer science curriculum in Wales within an
interdisciplinary Science \& Technology AoLE and supported by a
cross-curricular Digital Competence Framework (DCF) which is to be
implemented at all primary and secondary school years. Our case study
is contextualised by the relevant educational, economic and cultural
drivers for a small, bilingual nation. Following the phased
introduction of the DCF in 2016, we provide preliminary reflections on
its impact, and present the opportunities and challenges of the wider
practitioner-led curriculum reform process. We identify a number of
recommendations from across policy, pedagogy and practice, and its
potential replicability in other jurisdictions attempting similar
large-scale computer science and digital curriculum reforms.
\end{abstract}

\keywords{Computer science education, Digital competencies, STEM
education, Curriculum reform, Co-construction, Wales}

\maketitle

% From CfP:
% Short papers (up to 5 pages) focus on dissemination and discussion
% of new ideas in computing education practice or research that merit
% wider awareness and discussion within the community. They can present
% preliminary results of new educational innovations, present and
% discuss novel educational technologies, report work-in-progress
% research (including promising systems or tools that have not yet been
% evaluated and/or adopted extensively), or raise issues of significance
% for the development of the discipline, such as long-term strategic
% needs for computing education and curricula. All short papers are
% expected to have an appropriate coverage of literature to support the
% ideas and arguments that they present. Because it lacks some elements
% of a research paper, a short paper is evaluated mainly by its
% anticipated impact on discussions during the conference and possible
% future contribution to the field of computing education.

% check refs from JCE paper -- esp. Wales/UK ones...

\section{Introduction}

The structure of the paper is as follows...

% can we get away with a short section like in JCE?
\section{A Brief History of Computer Science Education in the UK}

Key citations~\cite{crick+sentance:2011,brown-et-al:sigcse2013,brown-et-al:toce2014,tryfonas+crick:petra2018}

% better title for this section?
\section{Rethinking the Digital Curriculum Wales}

% need to be careful about duplication of this section from JCE...
% maybe not have this as loads of sub-sections like the JCE paper --
% shorter, tighter narrative and then push more into the S&T and later sections.
\subsection{The Welsh Education System}

Having outlined the state of education in the UK as a whole, we now
restrict our attention to Wales. As our aim is to identify the
requirements for effective curriculum reform in a nation or region
which enjoys the same characteristics and challenges as Wales, we
start by describing these characteristics. We then outline the recent
history of education and curriculum reform with particular emphasis on
ICT and computing.

% ditto, but this will be expanded with more detail
\subsection{Review of the ICT Curriculum}

In January 2013, the Welsh Government initiated a review to consider
the future of computer science and ICT in schools in Wales. Its
primary thesis was that ICT in schools needed to be re-branded,
re-engineered and made relevant to now and to the future, with
computer science being introduced at primary school and developed over
the course of the curriculum so that learners can progress into a
career pathway in the sector; relevant skills, such as creative
problem-solving, should be explicitly reflected in the curriculum; and
revised qualifications should be developed in partnership with
schools, higher education and industry.

Key citation~\cite{wgictreview:2013}

% key themes to add context for previous and next two sections
\subsection{Review of the National Curriculum}

In March 2014, Professor Graham Donaldson, a former chief inspector of
schools in Scotland, was appointed by the Welsh Government to conduct
an independent review of curriculum and assessment of the entire
curriculum in Wales. This followed on from a number of previous
national-level consultations and reviews, including the 2013 review of
the ICT curriculum.

% DCF as a CCR, etc
% four main strands
% "rethinking ICT"
\subsection{A Digital Competence Framework}

Reflecting the importance of digital skills, the review added digital
competency as a new third cross-curricular responsibility, with
literacy and numeracy.

List the main strands and sub-strands, ref docs and justification,
etc. Headline outcome from the 

% rethinking "digital" and ICT

% ...but what about CS
% Science & Technology AoLE
\subsection{A New Approach for Computer Science}
% Phys
% Chem
% Biol

Key citations~\cite{crick+moller:wipsce2015,moller+crick:jce2018}

% what has happened to date, where are we in the journey?
\section{Progress So Far}

% NNEST and NNEM
% challenges and opportunities
% misunderstanding of "digital" and perceptions of ICT, etc --
% e.g. replacing ICT curric. with DCF. Cross-curric challenges.
% qualifications (e.g. ICT review), assessment, progression
% ITE, professional development, etc
% can this transfer to other jurisdictions? lesson learnt?
\section{Replicability and Looking Ahead}
% DCF 10 years in the making, inside context of curriculun reform, S&T

When establishing a model for viewing school computer science
education, it is apparent that there is substantial diversity between
school education systems (Snyder, 2012), and this can create obstacles
when trying to understand progress made in one country and potentially
replicate it in another (Hubwieser, 2013); this is also pertinent to
the devolved (and diverging) educational systems of the UK.

% main messages:

% 1) Digital competences on par with literacy and numeracy;
% 2) CS in an S&T context -- fourth science, link back to ICT review;
% 3) Networks of Excellence;

% 4) Practitioner-led, Pioneer model;

% 5) ITE, recruitment and retention
% professional development
% Reform fatique and skepticism, culture, geography
% Research-informed, pedagogy and practice
% Balancing wider economic context -- digital economy


%\section{Acknowledgements}

\bibliographystyle{ACM-Reference-Format}
\bibliography{SIGCSE2019} 

\end{document}
