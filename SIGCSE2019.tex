%%%% Proceedings format for most of ACM conferences (with the exceptions listed below) and all ICPS volumes.
\documentclass[sigconf]{acmart}
\usepackage{paralist}
\usepackage{url}
\usepackage[hyphenbreaks]{breakurl}

\def\UrlBreaks{\do\/\do-}

%%%% As of March 2017, [siggraph] is no longer used. Please use sigconf (above) for SIGGRAPH conferences.

%%%% Proceedings format for SIGPLAN conferences 
% \documentclass[sigplan, anonymous, review]{acmart}

%%%% Proceedings format for SIGCHI conferences
% \documentclass[sigchi, review]{acmart}

\usepackage{booktabs} % For formal tables


% Copyright
%\setcopyright{none}
%\setcopyright{acmcopyright}
\setcopyright{acmlicensed}
%\setcopyright{rightsretained}
%\setcopyright{usgov}
%\setcopyright{usgovmixed}
%\setcopyright{cagov}
%\setcopyright{cagovmixed}

\copyrightyear{2019}
\acmYear{2019}
\setcopyright{acmlicensed}
\acmConference[SIGCSE '19]{The 50th ACM Technical Symposium on
  Computer Science Education}{Feb. 27--Mar 2, 2019}{Minneapolis, MN, USA}
%\acmBooktitle{}
%\acmPrice{15.00}
%\acmDOI{10.1145/3159450.3159547}
%\acmISBN{978-1-4503-5103-4/18/02}
% This slight change to the code may also save 1 or 2 lines of space.

% removes the headers from each page per the preparation instructions, as these are not needed and will be updated with the chairs' actual session names during the pagination/indexing process:
\fancyhead{}

\begin{document}
\title{Computing and Digital Competence Within a Radical Curriculum Reform in Wales}
%\titlenote{}
%\subtitle{Extended Abstract}
%\subtitlenote{}

\author{Tom Crick}
\orcid{0000-0001-5196-9389}
\affiliation{%
  \institution{Swansea University}
  \streetaddress{}
  \city{Swansea} 
  \country{United Kingdom}
}
\email{thomas.crick@swansea.ac.uk}

\author{Faron Moller}
\orcid{0000-0001-9535-8053}
\affiliation{%
  \institution{Swansea University}
  \streetaddress{}
  \city{Swansea} 
  \country{United Kingdom}
}
\email{f.g.moller@swansea.ac.uk}
 
% The default list of authors is too long for headers}
%\renewcommand{\shortauthors}{Crick and Moller}


\begin{abstract}
We are witnessing significant computer science curriculum reform
across a number of regions and nations. Such reform in the United
Kingdom has been subject to intense scrutiny, particularly since
England introduced tis new computing curriculum in September 2014. In
Wales, we are starting to see the implementation and potential impact
of a radical new co-constructed, practitioner-led and purpose-driven
curriculum, with digital competencies as a core cross-curricular
responsibility distinct from computer science. As part of this reform,
the new curriculum in Wales is to be organised around six ``areas of
learning and experience'' (AoLEs), with a new Science \& Technology
AoLE bringing together the traditional physical sciences (physics,
chemistry and biology) along with computer science and design \&
technology.

In this paper, we present a developing national case study based
around the new computer science curriculum in Wales within an
interdisciplinary Science \& Technology AoLE and supported by a
cross-curricular Digital Competence Framework (DCF) which is to be
implemented at all primary and secondary school years. Our case study
is contextualised by the relevant educational, economic and cultural
drivers for a small, bilingual nation. Following the phased
introduction of the DCF in 2016, we provide preliminary reflections on
its impact, and present the opportunities and challenges of the wider
practitioner-led curriculum reform process. We identify a number of
recommendations from across policy, pedagogy and practice, and its
potential replicability in other jurisdictions attempting similar
large-scale computer science and digital curriculum reforms.
\end{abstract}

\keywords{Computer science education, Digital competencies, STEM
education, Curriculum reform, Co-construction, Wales}

\maketitle

% From CfP:
% Short papers (up to 5 pages) focus on dissemination and discussion
% of new ideas in computing education practice or research that merit
% wider awareness and discussion within the community. They can present
% preliminary results of new educational innovations, present and
% discuss novel educational technologies, report work-in-progress
% research (including promising systems or tools that have not yet been
% evaluated and/or adopted extensively), or raise issues of significance
% for the development of the discipline, such as long-term strategic
% needs for computing education and curricula. All short papers are
% expected to have an appropriate coverage of literature to support the
% ideas and arguments that they present. Because it lacks some elements
% of a research paper, a short paper is evaluated mainly by its
% anticipated impact on discussions during the conference and possible
% future contribution to the field of computing education.

\section{Introduction}

% can we get away with a short section like in JCE?
\section{A Brief History of Computer Science Education in the UK}

% better title for this section?
\section{Digital Curriculum Reform in Wales}

% need to be careful about duplication of this section from JCE...
\subsection{The Welsh Education System}

% ditto, but this will be expanded with more detail
\subsection{Review of the ICT Curriculum}

% key themes to add context for previous and next two sections
\subsection{Review of the National Curriculum}

% DCF as a CCR, etc
% four main strands
% "rethinking ICT"
\subsection{A Digital Competence Framework}

% ...but what about CS
% Science & Technology AoLE
\subsection{A New Approach for Computer Science}

% what has happened to date, where are we in the journey?
\section{Progress So Far}

% challenges and opportunities
% qualifications, assessment, progression
% ITE, professional development, etc
% can this transfer to other jurisdictions? lesson learnt?
\section{Replicability and Looking Ahead}

%\section{Acknowledgements}

\bibliographystyle{ACM-Reference-Format}
\bibliography{SIGCSE2019} 

\end{document}
