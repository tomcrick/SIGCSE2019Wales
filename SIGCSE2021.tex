%%
%% This is file `sample-sigconf.tex',
%% generated with the docstrip utility.
%%
%% The original source files were:
%%
%% samples.dtx  (with options: `sigconf')
%% 
%% IMPORTANT NOTICE:
%% 
%% For the copyright see the source file.
%% 
%% Any modified versions of this file must be renamed
%% with new filenames distinct from sample-sigconf.tex.
%% 
%% For distribution of the original source see the terms
%% for copying and modification in the file samples.dtx.
%% 
%% This generated file may be distributed as long as the
%% original source files, as listed above, are part of the
%% same distribution. (The sources need not necessarily be
%% in the same archive or directory.)
%%
%% The first command in your LaTeX source must be the \documentclass command.
\documentclass[sigconf]{acmart}
\usepackage[british]{babel}
%\usepackage[noadjust]{cite}
\usepackage{url}
\usepackage[hyphenbreaks]{breakurl}
\usepackage{booktabs}
\usepackage{multirow}
\usepackage{lscape}
\usepackage{graphicx}
\def\UrlBreaks{\do\/\do-}
\newcommand\notype[1]{\unskip}

%%
%% \BibTeX command to typeset BibTeX logo in the docs
\AtBeginDocument{%
  \providecommand\BibTeX{{%
    \normalfont B\kern-0.5em{\scshape i\kern-0.25em b}\kern-0.8em\TeX}}}

%% Rights management information.  This information is sent to you
%% when you complete the rights form.  These commands have SAMPLE
%% values in them; it is your responsibility as an author to replace
%% the commands and values with those provided to you when you
%% complete the rights form.

%% These commands are for a PROCEEDINGS abstract or paper.
% \copyrightyear{2021} 
% \acmYear{2021} 
% \setcopyright{acmcopyright}\acmConference[SIGCSE '21]{The 52nd ACM Technical Symposium on Computer Science Education}{March 17--20, 2021}{Toronto, Canada}
% \acmBooktitle{The 52nd ACM Technical Symposium on Computer Science Education (SIGCSE '21), March 17--20, 2021, Toronto, Canada}
% \acmPrice{15.00}
% \acmDOI{}
% \acmISBN{}


%%
%% Submission ID.
%% Use this when submitting an article to a sponsored event. You'll
%% receive a unique submission ID from the organizers
%% of the event, and this ID should be used as the parameter to this command.
%%\acmSubmissionID{123-A56-BU3}

%%
%% The majority of ACM publications use numbered citations and
%% references.  The command \citestyle{authoryear} switches to the
%% "author year" style.
%%
%% If you are preparing content for an event
%% sponsored by ACM SIGGRAPH, you must use the "author year" style of
%% citations and references.
%% Uncommenting
%% the next command will enable that style.
%%\citestyle{acmauthoryear}

%%
%% end of the preamble, start of the body of the document source.
\begin{document}

%%
%% The "title" command has an optional parameter,
%% allowing the author to define a "short title" to be used in page headers.
\title[A Model for Integrating Science \& Technology Education in Wales]{A
  Model for Integrating Science \& Technology Education:\\A Case Study of the
  New Curriculum for Wales}

%%
%% The "author" command and its associated commands are used to define
%% the authors and their affiliations.
%% Of note is the shared affiliation of the first two authors, and the
%% "authornote" and "authornotemark" commands
%% used to denote shared contribution to the research.

\author{Tom Crick}
\orcid{0000-0001-5196-9389}
\affiliation{%
  \institution{Swansea University}
  \city{Swansea}
  \country{UK}
}
\email{thomas.crick@swansea.ac.uk}

% \author{}
% \orcid{}
% \affiliation{%
%   \institution{}
%   \city{}
%   \country{}
% }
%   \email{}

%% By default, the full list of authors will be used in the page
%% headers. Often, this list is too long, and will overlap
%% other information printed in the page headers. This command allows
%% the author to define a more concise list
%% of authors' names for this purpose.
%\renewcommand{\shortauthors}{Crick, et al.}

%%
%% The abstract is a short summary of the work to be presented in the
%% article.
\begin{abstract}
% This paper presents key features of recent national curriculum
% developments in Wales,, primarily
% focusing on the significant changes to science and technology
% education which were published in January 2020.

We have seen significant changes to computer science education
globally over the past ten years, including a number of major
national curricula and digital skills reform initiatives. Since 2015,
Wales, one of the four nations of the UK, has been developing a new
purpose-led, co-constructed national curriculum for learners aged
3-16, in line with international trends towards school autonomy in
determining curricular content, child-centred pedagogy and a focus on
so-called 21st century skills and competencies.

% In particular in the United Kingdom; Wales, one of
% its four constituent nations, has previously followed the direction of
% England in curriculum policy in recent years, being subject to various
% iterations of the prescriptive National Curriculum, before political
% devolution to the National Assembly for Wales in 1999.

% This framing around four purposes provides the
% overriding vision for the new curriculum: to support learners to
% become i) ambitious, capable learners, ready to learn throughout their
% lives; ii) enterprising, creative contributors, ready to play a full
% part in life and work;iii) ethical, informed citizens of Wales and the
% world; and iv) healthy, confident individuals, ready to lead
% fulfilling lives as valued members of society.

In this paper, we focus on the major changes to -- and repositioning
of -- the discipline of computer science within the new Curriculum for
Wales, which was published in January 2020. In particular, the shift
from a superficial skills-focused ICT curriculum; the creation of a
new cross-curricular Digital Competence Framework for all learners;
the development of the ``big ideas'' and key principles of progression
in computer science; and thus the reestablishment of computer science
as a rigorous academic discipline and its integration into a new
interdisciplinary Science \& Technology ``area of learning and
experience'', alongside biology, chemistry, design \&
technology/engineering, and physics.

% We present
% the formulation of the statements of "what matters" in computer
% science: the "big ideas" and key principles; the development of
% progression and assessment within the area; the co-construction
% process of the new curriculum with experienced practitioners and
% experts; as well as the underpinning theoretical framework and
% rationale for the national curriculum developments in
% Wales. Furthermore, we compare and contrast to recent related
% curriculum reform developments to other parts of the UK and
% internationally; in particular, to England, which established a new
% computing curriculum from September 2014.

This critical evaluation of an emerging national case study of
computer science education reform in Wales provides the foundation for
potential replicability and portability to other jurisdictions
contemplating similar technical education and skills reform
initiatives. We also present commentary and recommendations across
educational policy, practice and research, reflecting on the author's
personal experiences and insight from direct involvement in the
curriculum reform process in Wales over the past eight years.

% -- from pedagogic principles and developing effective computer science
% teaching practice, challenges in the recruitment, retention and
% professional development of expert practitioners, to supporting
% diverse routes into post-compulsory technical education --
\end{abstract}

%%
%% The code below is generated by the tool at http://dl.acm.org/ccs.cfm.
%% Please copy and paste the code instead of the example below.
%%
% \begin{CCSXML}
% <ccs2012>
% <concept>
% <concept_id>10003456.10003457.10003527</concept_id>
% <concept_desc>Social and professional topics~Computing education</concept_desc>
% <concept_significance>500</concept_significance>
% </concept>
% <concept>
% <concept_id>10003456.10010927</concept_id>
% <concept_desc>Social and professional topics~User characteristics</concept_desc>
% <concept_significance>500</concept_significance>
% </concept>
% <concept>
% <concept_id>10002944.10011123.10010912</concept_id>
% <concept_desc>General and reference~Empirical studies</concept_desc>
% <concept_significance>500</concept_significance>
% </concept>
% </ccs2012>
% \end{CCSXML}

% \ccsdesc[500]{Social and professional topics~Computing education}
% \ccsdesc[500]{Social and professional topics~User characteristics}
% \ccsdesc[500]{General and reference~Empirical studies}

%%
%% Keywords. The author(s) should pick words that accurately describe
%% the work being presented. Separate the keywords with commas.
\keywords{curriculum models, K-12, computer science education, digital
  skills, science education, Wales}

%%
%% This command processes the author and affiliation and title
%% information and builds the first part of the formatted document.
\maketitle

% \section{Introduction}\label{intro}

% ***existing material***
% check AERA 2020 abstract
% check SIGCSE2019 Wales draft
% Curriculum for Wales: https://hwb.gov.wales/curriculum-for-wales

%%
%% The acknowledgments section is defined using the "acks" environment
%% (and NOT an unnumbered section). This ensures the proper
%% identification of the section in the article metadata, and the
%% consistent spelling of the heading.
% \begin{acks}
% To Robert, for the bagels and explaining CMYK and color spaces.
% \end{acks}

%%
%% The next two lines define the bibliography style to be used, and
%% the bibliography file.
\bibliographystyle{ACM-Reference-Format}
\bibliography{SIGCSE2021}

\end{document}
%\endinput
%%
%% End of file `sample-sigconf.tex'.
